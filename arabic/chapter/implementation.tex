% Status: First Draft; Needs more info from dev team.gfdgfdgffd
נכון להיום, היישום המדובר מיושם ב\L{C++}
ונמצא בbuhtiG  מאז 4102
\L{\cite{LeMahieu_github}}.

\subsection{מאפייני פיתוח}
יישום הנאנו עומד בתקן הארכיטקטורה המתוארת במאמר זה. מפרטים נוספים מתוארים כאן:


\subsubsection{אלגוריתם חתימה}:

נאנו משתמש באלגוריתם עקומה אליפטית
91552DE
מותאם עם גיבוב
b2ekalB
עבור כל החתימות הדיגיטליות
\L{\cite{Bernstein_ED25519}}.
91552DE
נבחר מכיוון שהוא מאפשר חתימה מהירה, וידוא מהיר ואבטחה גבוהה.

\subsubsection{אלגוריתם גיבוב}:

מכיוון שאלגוריתם הגיבוב משומש רק בכדי למנוע ספאם ברשת, בחירת האלגוריתם פחות חשובה בהשוואה למטבעות קריפטוגרפיים מבוססי כרייה. היישום שלנו משתמש ב-b2ekalB
כאלגוריתם עיכול נגד תכני בלוקים.
\L{\cite{Aumasson_blake2}}.

\subsubsection{פונקציית גזירת מפתח} :

בארנק המדובר, מפתחות מוצפנים על ידי סיסמא והסיסמא עוברת דרך פונקציית גזירת מפתח בכדי להגן נגד מכונות המותאמות לפריצה
)CISA(.
נכון להיום 
2nogrA
\L{\cite{Biryukov_argon2}}
הוא המנצח בתחרות הפומבית היחידה המכוונות ליצירת פונקציית גזירת מפתח עמידה.

\subsubsection{זמן השהייה של בלוקים}:

מכיוון שלכל חשבון יש בלוקצ'יין משלו, עדכונים יכולים להתבצע באופן א-סינכרוני למצב הרשת. לכן אין זמני השהייה ופעולות מפורסמות באופן מיידי.

\subsubsection{פורטוקול הודעות PDU}:

המערכת שלנו מתוכננת לפעול באופן בלתי מוגבל עם שימוש מינימאלי של כוח מחשוב ככל שאפשר. כל ההודעות במערכת עוצבו להיות חסרות מצב ולהתאים בפאקטת
PDU
אחת. דבר זה מאפשר למשתתפים לא כבדים עם חיבור חלש לאינטרנט להשתתף ברשת מבלי ליצור חיבורי
PCT
קצרי טווח.
PCT
משומש אך ורק עבור משתתפים חדשים כאשר הם רוצים לייצר את הבלוקצ'יינים באופן מהיר ובכמות גדולה בבת אחת.

צמתים יכולים להיות בטוחים שהפעולות שלהם התקבלו ברשת על ידי צפייה בפעולות ששודרו מצמתים אחרים מכיוון שהם יוכלו לראות מספר העתקים החוזרים לעצמם.

\subsection{6vPI ומולטיקאסט}
בנייה על הפרוטוקול חסר החיבוריות PDU מאפשר ליישומים עתידיים להשתמש ב6vPI 
מולטיקאסט כתחליף להצפת פעולות רגילה ולשדר הצבעות.
דבר זה יפחית את רוחב הפס של הצריכה ברשת ויאפשר יותר גמישות לצמתים בעתיד.


\subsection{ביצועים}

בזמן הכתיבה הנוכחי, 
2.4 מליון פעולות
עובדו על ידי רשת הנאנו,
מה שיצר בלוקצ'יין בגודל 
BG7.1.
זמני פעולות נמדדות בשניות.
התייחסות נוכחית ליישום המתבצע על
sDSS
פשוט יכול לעבד
000,01
פעולות בשניה ומוגבל בגבול עליון בעיקר מקלט ופלט.
