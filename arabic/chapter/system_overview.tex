% Status: First Draft; Ready for Review
בניגוד לבלוקצ'יינים המשומשים במטבעות קריפטוגרפים אחרים, נאנו משתמשת במבנה של סריג-בלוקים. לכל חשבון יש בלוקצ'יין משלו )שרשרת-חשבון( המקבילה להסטוריית הפעולות/מאזן של החשבון )איור 2(. כל שרשרת-חשבון יכולה להתעדכן רק על ידי בעל החשבון; דבר זה מאפשר לכל שרשרת-חשבון להתעדכן באופן מיידי וא-סינכרוני לשאר הסריג-בלוקים, מה שיוצר פעולות מהירות. הפרוטוקול של נאנו הוא קל משקל בצורה משמעותית; כל פעולה נכנסת בגודל המיניאלי הדרוש לפאקטת PDU לצורך העברה באינטרנט. דרישות החומרה לצמתים הן גם מינימאליות, מכיוון שצמתים צריכים רק להקליט ולשדר בלוקים לרוב הפעולות )איור 1(.

המערכת מאותחלת עם חשבון ראשוני   המכיל את המאזן הראשוני. המאזן הראשוני הוא בעל כמות קבועה ולעולם לא יוכל לגדול. המאזן הראשוני מחולק ונשלח לחשבונות אחרים על ידי העברת פעולות הרשומות על השרשרת-חשבון של החשבון הראשוני. סכום המאזנים של כל החשבונות לעולם לא יעבור את מאזן החשבון הראשוני, דבר הנותן למערכת גבול עליון על כמות וללא יכולת להגדיל אותה.

חלק הזה יעבור על איך סוגי פעולות שונות נבנות ומועברות דרך הרשת.

\begin{figure}[!ht]
\L {
   \centering
   \begin{tikzpicture}[node distance=1cm]
      \node (a) [account_name]{A};
      \node (b) [account_name, right = of a, xshift=0.75cm]{B};
      \node (c) [account_name, right = of b, xshift=0.75cm]{C};
            
      \node (c_1) [t_circ, above = of c, yshift=0cm]{\R{ש}};
      \node (c_2) [t_circ, above = of c_1, yshift=0cm]{\R{ק}};
      \node (c_3) [t_circ, above = of c_2, yshift=0cm]{\R{ק}};
      \node (c_4) [t_circ, above = of c_3, yshift=0cm]{\R{ק}};
      
      \node (a_1) at (c_2-| a)[t_circ]{\R{ש}};
      \node (a_2) [t_circ, above = of a_1, yshift=0cm]{\R{ק}};
      \node (a_3) [t_circ, above = of a_2, yshift=0]{\R{ק}};
            
      \node (b_1) at (c_1 -| b) [t_circ]{\R{ש}};
      \node (b_2) at (c_3 -| b) [t_circ]{\R{ש}};
      \node (b_3) [t_circ, above = of b_2]{\R{ש}};
            
      \node (a_ellipsis) [inv_account_name, above=of a_3]{$\rvdots$};
      \node (b_ellipsis) at (a_ellipsis -| b) [inv_account_name]{$\rvdots$};
      \node (c_ellipsis) at (a_ellipsis -| c) [inv_account_name]{$\rvdots$};

      \draw [line] (a) -- (a_1);
      \draw [line] (a_1) -- (a_2);
      \draw [line] (a_2) -- (a_3);
      \draw [arrow] (a_3) -- (a_ellipsis);
      
      \draw [line] (b) -- (b_1);
      \draw [line] (b_1) -- (b_2);
      \draw [line] (b_2) -- (b_3);
      \draw [arrow] (b_3) -- (b_ellipsis);
      
      \draw [line] (c) -- (c_1);
      \draw [line] (c_1) -- (c_2);
      \draw [line] (c_2) -- (c_3);
      \draw [line] (c_3) -- (c_4);
      \draw [arrow] (c_4) -- (c_ellipsis);
      
      \draw [dashed_arrow] (c_1) -- (a_2);
      \draw [dashed_arrow] (b_1) -- (c_3);
      \draw [dashed_arrow] (b_3) -- (c_4);
      \draw [dashed_arrow] (a_1) -- (c_2);
      \draw [dashed_arrow] (b_2) -- (a_3);
      
      \node (time)[inv_account_name, rotate=90, left=of a_1]{\R{זמן}};
      \node (e_time)[inv_account_name, left=of a_2, rotate=90, xshift=0.5cm]{};
      
      \draw [arrow] (time) -- (e_time);
   \end{tikzpicture}
   \R{\caption{המחשה של סריג-הבלוקים. כל העברה של כספים דורשת בלוק שליחה )ש( ובלוק קבלה )ק(, כאשר כל בלוק נחתם על ידי בעל שרשרת החשבון )C ,B ,A( }}
   \label{fig:Aross_account_chain}}
\end{figure}

\subsection{פעולות} \label{sec:transactions}
העברת כספים מחשבון אחד לאחר דורשת שתי פעולות, פעולת שליחה המפחיתה את הכמות ממאזן השולח ופעולת קבלה המוסיפה את הכמות למאזן המקבל )איור 3(.

העברת כמויות כפעולות שונות בחשבונות השולח והמקבל משמשת עבור מספר מטרות:
\begin{enumerate}
   \item סידור פעולות מתקבלות שהן א-סינכרוניות באופן טבעי.
   \item שמירה על פעולות קטנות בכדי שיוכלו להתאים לפאקטת PDU.
   \item מאפשרת גיזום פנקס החשבונות על ידי הקטנת עקבת המידע.
   \item בידוד של פעולות מאושרות מפעולות שאינן מאושרות.
\end{enumerate}

יותר מחשבון אחד שמעביר לאותו חשבון יעד היא פעולה א-סינכרונית. זמן השהיית הרשת והעובדה שהחשבונות המקבלים לא בהכרח בתקשורת אחד עם השני אומר שאין דרך אוניברסלית מקובלת לדעת איזה פעולה התבצעה קודם. מכיוון שחיבור הינה פעולה אסוציאטיבית, סדר קבלת הפעולות אינו משנה, ולכן אנחנו פשוט צריכים הסכמה כללית. זהו מרכיב עיקרי שמעביר הסכמה בזמן-ריצה להסכמה בזמן-ארכיטקטורה. לחשבון המקבל יש שליטה על ההחלטה איזה העברה הגיעה קודם והיא מבוטאת על ידי סדר הבלוקים המתקבלים החתומים. 

אם חשבון רוצה לבצע העברה גדולה שהתקבלה כקבוצת העברות קטנות, נרצה להציג זאת בדרך שמתאימה לפקטת PDU אחת. כאשר חשבון מקבל מסדר העברות שהתקבלו, הוא שומר על סכום מאזן החשבון שלו כך שבכל זמן נתון, יש לו את היכולת להעביר כל כמות בעזרת פעולה בגודל קבוע. דבר זה שונה ממבנה פעולות יוצאות/נכנסות אשר משומש בביטקוין ובמטבעות קריפטוגרפים אחרים. 

חלק מהצמתים אינם מעוניינים בהרחבת משאבים לצורך שמירה על הסטורית הפעולות המלאה של חשבון; הם רק מעוניינים במאזן הנוכחי של כל חשבון. כאשר חשבון מבצע פעולה, הוא מקודד את המאזן הנצבר שלו וצמתים אלו רק צריכים לעקוב אחרי הבלוק האחרון, אשר מאפשר להם להתעלם מהיסטוריית המידע ועדיין לשמור על נכונות.

אפילו עם דגש על הסכמי זמן-ריצה, ישנה השהייה כאשר מאשרים פעולות בגלל הצורך לזיהוי וטיפול בגורמים זדוניים ברשת. מכיוון שהסכמים בנאנו מגיעים מהר, בסדר של מילי-שניות עד שניות, נוכל להציג למשתמש שני קטגוריות מוכרות של פעולות מגיעות: מאושרות ולא מאושרות. פעולות מאושרות הן פעולות שבהן חשבון יצר בלוקי קבלה. פעולות לא מאושרות עדיין לא הוטמעו במאזן של מקבל הפעולה. זהו תחליף לאמות המדידה היותר מסובכות והלא מוכרות שבהן מטבעות אחרים משתמשים.

\subsection{יצירת חשבון}\label{sec:open}
כדי ליצור חשבון, צריך להוציא פעולת פתיחה )איור 4(. פעולת פתיחה היא תמיד הפעולה הראשונה בכל שרשרת-חשבון ויכולה להיווצר בקבלה הראשונה של כספים. שדה ה-\textit{tnuocca} שומר את המפתח הפומבי )הכתובת( שנוצר מתוך המפתח הפרטי שמשומש בחתימה. שדה ה-\textit{ecruos} מכיל את הגיבוב של הפעולה ששלחה את הכספים. בזמן יצירת חשבון, נציג חייב להיבחר בשביל להצביע בשמך; ניתן לשנות נציג בשלב מאוחר יותר )חלק~\ref{sec:change}(. החשבון יכול להכריז על עצמו כנציג של עצמו.

\begin{figure}[!ht]
\L{
\begin{lstlisting}
open {
   account: DC04354B1...AE8FA2661B2,
   source: DC1E2B3F7C...182A0E26B4A,
   representative: xrb_1anr...posrs,
   work: 0000000000000000,
   type: open,
   signature: 83B0...006433265C7B204
}
\end{lstlisting}

\R{\caption{אנטומיה של פעולת פתיחה}}
\label{code:open}
}
\end{figure}

\subsection{מאזן חשבון}\label{sec:account_balance}
מאזן החשבון נשמר בתוך פנקס החשבונות עצמו. במקום  לשמור את סכומי הפעולות, אישור )חלק~\ref{sec:transaction_verification}( דורש בדיקה של ההפרש בין המאזן בבלוק השליחה למאזן בבלוק הקודם לו. החשבון המקבל יוכל להגדיל את את המאזן הקודם לתוך המאזן הסופי שניתן בבלוק הקבלה החדש. זה נעשה בכדי לשפר את מהירות העיבוד כאשר מורידים סכומים גדולים של בלוקים. כאשר מבקשים את הסטורית החשבון, סכומים ניתנים מיידית.

\subsection{שליחה מחשבון} \label{sec:send}
כדי לשלוח מכתובת, לכתובת חייבת להיות בלוק פתיחה קיים, ולכן מאזן )איור 5(. שדה ה-suoiverp מכיל את הגיבוב של הבלוק הקודם בשרשרת-החשבון. שדה ה-noitanitsed מכיל את החשבון לשליחת הכספים. בלוק שליחה אינו ניתן לשינוי ברגע שהוא מאושר. ברגע שהבלוק משודר אל הרשת, הכספים מופחתים מיידית מהמאזן של חשבונות השולחים ומחכים כ"ממתינים" עד שהצד המקבל חותם על בלוק שמקבל את הכספים. כספים ממתינים לא נחשבים ככספים ה"מחכים לאישור" מכיוון שמצד השולחים, הכספים האלו נעלמו ולא יכולים לחזור.

\begin{figure}[!ht]
\L {
\begin{lstlisting}
send {
   previous: 1967EA355...F2F3E5BF801,
   balance: 010a8044a0...1d49289d88c,
   destination: xrb_3w...m37goeuufdp,
   work: 0000000000000000,
   type: send,
   signature: 83B0...006433265C7B204
}
\end{lstlisting}
\R{\caption{אנטומיה של פעולת שליחה}}
\label{code:send}
}
\end{figure}

\subsection{קבלת פעולה}\label{sec:receive}
כדי להשלים פעולה, המקבל של הכספים חייב ליצור בלוק קבלה בחשבון-השרשרת שלו )איור 6(. שדה ה-ecruos מצביע על גיבוב בלוק השליחה המתאים. ברגע שבלוק זה נוצר ומשודר לרשת, מאזני החשבונות מעודכנים והכספים עברו באופן רשמי לחשבון המקבל.

\begin{figure}[!ht]
\L {
\begin{lstlisting}
receive {
   previous: DC04354B1...AE8FA2661B2,
   source: DC1E2B3F7C6...182A0E26B4A,
   work: 0000000000000000,
   type: receive,
   signature: 83B0...006433265C7B204
}
\end{lstlisting}
\R{\caption{אנטומיה של פעולת קבלה}}
\label{code:receive}
}
\end{figure}

\subsection{בחירת נציג}\label{sec:change}
העובדה שמחזיקי חשבונות יכולים לבחור נציג שיצביע בשמם היא כלי ביזור חזק שאין לו השוואה בפרוטוקולי הוכחת עבודה והוכחת החזקה. במערכות הוכחת החזקה קונבנציונאליות, צומת בעל החשבון חייבת לרוץ כדי להשתתף בהצבעה. הרצה של צומת באופן מתמשך אינה פרקטית להמון משתמשים; נתינת הכוח להצביע לנציג בשם החשבון מרגיעה דרישה זו. לבעלי החשבון ניתנת האפשות להעביר הסכמה לכל חשבון בכל רגע נתון. פעולת שינוי משנה את נציג החשבון על ידי הורדת משקל ההצבעה מהנציג הקודם והעברתו לנציג החדש )איור 7(. כספים אינם מועברים בפעולה זו ולנציג אין שליטה על הכספים של מחזיק החשבון.

\begin{figure}[!ht]
\L {
\begin{lstlisting}
change {
   previous: DC04354B1...AE8FA2661B2,
   representative: xrb_1anrz...posrs,
   work: 0000000000000000,
   type: change,
   signature: 83B0...006433265C7B204
}
\end{lstlisting}
\R{\caption{אנטומיה של פעולת שינוי}}
\label{code:change}
}
\end{figure}

\subsection{פיצולים והצבעות} \label{sec:forks}
פיצול קורה כאשר $j$ בלוקים חתומים $b_1, b_2, \dots, b_j$ טוענים שאותו בלוק הוא הקודם )איור \ref{fig:fork}(. בלוקים אלו גורמים למצב קונפליקט על הסטאטוס של החשבון וחייבים להיות מטופלים. רק לבעל החשבון יש את היכולת לחתום על בלוקים לתוך השרשרת-חשבון שלו, לכן פיצול חייב להיות תוצאה של תכנות לקוי או כוונה זדונית )בזבוז כספים כפול( של בעל החשבון.

\begin{figure}[!ht]
\L{
   \centering
   \begin{tikzpicture}[node distance=0.5cm]
      %%%%%%%%%%%%%
      % ACCOUNT A %
      %%%%%%%%%%%%%
      \node (account_A_0) [generic_node]
            {\R{חשבון A} \\ \R{בלוק $i$}};
      \node (account_A_1) [generic_node, left = of account_A_0]
            {\R{חשבון A} \\ \R{בלוק $i+1$}};
      \node (account_A_2a) [generic_node, left = of account_A_1, yshift=1cm] 
            {\R{חשבון A} \\ \R{בלוק $i+2$}};
      \node (account_A_2b) [generic_node, left = of account_A_1, yshift=-1cm]
            {\R{חשבון A} \\ \R{בלוק $i+2$}};
      
      \draw [arrow] (account_A_1) -- (account_A_0);
      \draw [arrow] (account_A_2a) -- (account_A_1);
      \draw [arrow] (account_A_2b) -- (account_A_1);
   \end{tikzpicture}
   \R{\caption{פיצול קורה כאשר שניים )או יותר( בלוקים חתומים מצביעים לאותו בלוק קודם. בלוקים ישנים יותר נמצאים מימין. בלוקים חדשים יותר נמצאים משמאל}
   \label{fig:fork}
   }}
\end{figure}

בעת גילוי, נציג יצור הצבעה עם קישור לבלוק $\hat{b}_i$ בפנקס החשבונות שלו וישדר אותה לרשת. משקל ההצבעה של הצומת, $w_i$  הוא סכום המאזנים של כל החשבונות שבחרו בו להיות נציג. הצומת תעקוב אחרי הצבעות מגיעות משאר $M$  נציגים ותשמור סכום מצטבר ל4 תקופות הצבעה, סך הכל כדקה אחת, ותאשר את הבלוק המנצח )משוואה 1(.

\begin{align}
   v(b_j) &= \sum_{i=1}^M w_i\mathbbm{1}_{\hat{b}_i=b_j} \label{eq:weighted_vote} \\
   b^* &= \argmax_{b_j} v(b_j) \label{eq:most_votes}
\end{align}

לבלוק הפופולארי ביותר $b^*$ יהיה את מרבית הקולות והוא ישמר בפנקס החשבונות של הצומת )משוואה 2(. הבלוק)ים( שהפסידו את ההצבעה ייעלמו. אם נציג מחליף בלוק בפנקס החשבונות שלו, הוא יצור הצבעה חדשה עם מספר רצף גבוה יותר וישדר את ההצבעה החדשה אל הרשת. זהו התרחיש היחידי בו נציג מצביע.

במקרים מסוימים, בעיות תקשורת קצרות יכולות למנוע את קבלת הבלוק ששודר על ידי שאר השותפים ברשת. משתתפים ברשת שלא ראו את השידור הראשוני יתעלמו מכל בלוק נלווה אחר בחשבון. שידור מחדש של הבלוק אל הרשת יתקבל על ידי שאר משתתפי הרשת ובלוקים אחרים יוחזרו באופן אוטומאטי. אפילו במצב של פיצול או בלוק חסר, רק החשבונות המשוייכים לפעולה מושפעים; שאר הרשת ממשיכה בעיבוד פעולות לשאר החשבונות.

\subsection{הוכחת עבודה} \label{sec:pow}
לכל ארבעת סוגי הפעולות יש שדה kroW שחייב להיות מאוכלס. שדה ה-kroW מאפשר ליוצר הפעולה לחשב מספר חד פעמי )ecnoN( כך שהגיבוב של מספר זה ביחד עם שדה ה-suoiverP בפעולות קבלה/שליחה/שינוי או שדה ה-tnuoccA  בפעולת פתיחה הינו מתחת לסף מסוים. בניגוד לביטקוין, הוכחת העבודה בנאנו הינה בשימוש כמנגנון נגד ספאם, בדומה לhsachsaH, וניתן לחשבו בקנה מידה של שניות \cite{Back_hashcash}. ברגע שפעולה נשלחת, הוכחת העבודה של בלוקים עוקבים יכולה להיות מחושבת מראש מכיוון ששדה ה-suoiverP כבר ידוע; דבר זה יגרום לפעולות להופיע באופן מיידי למשתמש הקצה כל עוד הזמן בין הפעולות יותר גדול מהזמן שנדרש כדי לחשב את הוכחת העבודה.

\subsection{אישור פעולה} \label{sec:transaction_verification}
כדי שבלוק יהיה נחשב כתקף, חייב להיות לו את התכונות הבאות:
\begin{enumerate}
   \item הבלוק לא יכול להיות קיים בפנקס החשבונות )פעולה כפולה(.
   \item חייב להיות חתום על ידי בעל החשבון.
   \item הבלוק הקודם הוא ראש הבלוק של השרשרת-חשבון. אם הוא קיים אבל הוא לא הראש, ישנו פיצול.
   \item לחשבון חייב להיות בלוק פתיחה.
   \item הגיבוב המחושב עובר את הסף שנדרש על ידי הוכחת העבודה.
\end{enumerate}
אם מדובר בבלוק קבלה, נבדוק אם גיבוב בלוק המקור הוא בסטאטוס ממתין, כלומר עדיין לא אושר. אם מדובר על בלוק שליחה, המאזן חייב להיות פחות מהמאזן של הבלוק הקודם.