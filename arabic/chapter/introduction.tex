% Status: First Draft; Ready for Review
\IEEEPARstart{מ}{אז} היישום של ביטקויין ב9002, החל מעבר גדול ממטבעות סטנדרטיים שנופקו על ידי הממשלה וממערכות פיננסיות לכיוון מערכות תשלום מודרניות מבוססות קריפטוגרפיה, אשר מציעות את האפשרות לאכסן ולהעביר כספים בצורה בטוחה ונטולת צורך באמון \cite{Nakamoto_bitcoin:a}. בכדי שיוכלו לפעול בצורה אפקטיבית, מטבע קריפטוגרפי חייב להיות קל להעברה, ללא אפשרות להחזרה ושיהיה בעל עמלה מזערית או ללא עמלה כלל. הזמן הנדרש לבצע תשלום, העמלות הגבוהות וחוסר העמידה בעומס העלו שאלות לגבי השימושיות היומיומית של הביטקוין כמטבע.

במאמר זה, נציג את נאנו, מטבע קריפטוגרפי בעל זמן השהייה נמוך הנבנה על מבנה נתונים חדשני בשם סריג-בלוקים המציע עמידה בעומס בלתי מוגבל ופעולות ללא עמלות. נאנו הוא פרוטוקול פשוט שמטרתו היחידה היא להיות מטבע קריפטוגרפי בעל ביצועים גבוהים. פרוטוקול הנאנו יכול לרוץ על חומרה בעלת עוצמה נמוכה, דבר המאפשר לו להיות מטבע קריפטוגרפי מבוזר פרקטי לשימוש יום יומי.

הסטטיסטיקות של המטבעות הקריפטוגרפים המוצגות במאמר הזה מדוייקות לזמן פרסום המאמר.
