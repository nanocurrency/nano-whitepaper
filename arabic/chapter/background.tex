% Status: First Draft; Ready for Review
% Even though this figure doesn't belong in this section, it's placed here for better placement in the rendered paper.
\begin{figure*}[!ht]
\L {
  \centering
  \subfloat[a][\R{כאשר לא מזוהת התנגשות, אין צורך בתוספת של תקורה.}]{
    \centering
    \begin{tikzpicture}[node distance=0.5cm]
      %%%%%%%%%%%%%%%
      % No Conflict %
      %%%%%%%%%%%%%%%
      \node (receive) [generic_node]
          {\R{קבלה}};
      \node (repeat) [generic_node, left = of receive]
          {\R{חזרה}};
      \node (observe) [generic_node, left = of repeat]
          {\R{צפייה}};
      \node (fake) [center_text, left = of observe]{};
      \node (quorum) [generic_node, left = of fake] 
          {\R{קוורום}};
      \node (confirm) [generic_node, left = of quorum]
          {\R{אישור}};
      
      \draw [arrow] (receive) -- (repeat);
      \draw [arrow] (repeat) -- (observe);
      \draw [arrow] (observe) -- (quorum);
      \draw [arrow] (quorum) -- (confirm);
      
    \end{tikzpicture}
  }
  \newline
  \subfloat[b][\R{במקרה של פעולות מתנגשות, צמתים מצביעים לפעולה המאושרת.}]{
    \centering
    \begin{tikzpicture}[node distance=0.5cm]
      %%%%%%%%%%%%
      % Conflict %
      %%%%%%%%%%%%
      \node (receive) [generic_node]
          {\R{קבלה}};
      \node (repeat) [generic_node, left = of receive]
          {\R{חזרה}};
      \node (observe) [generic_node, left = of repeat]
          {\R{צפייה}};
      \node (conflict) [generic_node, left = of observe]
          {\textbf{\R{התנגשות}}};
      \node (vote) [generic_node, left = of conflict] 
          {\R{הצבעה}};
      \node (confirm) [generic_node, left = of vote]
          {\R{אישור}};
      
      \draw [arrow] (receive) -- (repeat);
      \draw [arrow] (repeat) -- (observe);
      \draw [arrow] (observe) -- (conflict);
      \draw [arrow] (conflict) -- (vote);
      \draw [arrow] (vote) -- (confirm);
      
    \end{tikzpicture}
  }
  \R{\caption{
נאנו לא דורש תוספת תקורה לפעולה סטנדרטית. במקרה של פעולות מתנגשות, צמתים חייבים להצביע לפעולה שעליהם ישמרו.}}
  \label{fig:transaction_flow}
}
\end{figure*}


ב8002, אינדיבידואל אנונימי תחת שם העט סאטושי נאקאמוטו פרסם מאמר המציג את המטבע הקריפטוגרפי המבוזר הראשון בעולם, ביטקוין \cite{Nakamoto_bitcoin:a}. חידוש מרכזי שהביא הביטקוין הוא הבלוקצ'יין, מבנה נתונים מבוזר פומבי ולא ניתן לשינוי אשר משמש כפנקס חשבונות לתשלומי המטבע. לרוע המזל, ככל שביטקוין גדל, נחשפו מספר בעיות בפרוטוקול אשר יצרו בעייה עבור יישומים רבים:

\begin{enumerate}
    \item אי עמידה בעומס: כל בלוק בבלוקצ'יין יכול לאחסן כמות מוגבלת של מידע, מה שאומר שהמערכת יכולה לעבד כמות מוגבלת של פעולות בשנייה, מה שהופך את מספר המקומות בבלוק למשאב. נכון לעכשיו, חציון עמלת התשלום הינו \$83.01 \cite{Bitcoin_med_fee}.
    \item זמן השהייה גבוה: זמן אישור תשלום ממוצע הינו 461 דקות \cite{Bitcoin_avg_confirmation_time}.
    \item חוסר יעילות בחשמל: רשת הביטקוין צורכת כhWT82.72 בשנה, כלומר hWK062 עבור פעולה בממוצע \cite{Bitcoin_energy_index}.
\end{enumerate}

ביטקוין, ושאר מטבעות קריפטוגרפיים, פועלים על ידי השגת הסכמה על פנקסי החשבונות הגלובליים שלהם בשביל לאשר פעולות לגיטימיות תוך כדי התנגדות לגורמים זדוניים. ביטקוין משיג הסכמה על ידי דרך מדידה כלכלית הנקראת הוכחת עבודה )kroW fo foorP(. במערכת הוכחת עבודה משתתפים מתחרים בתחרות בה המטרה היא לחשב מספר חד פעמי )\textit{ecnon(}, כך שהגיבוב של כל בלוק נמצא בטווח מטרה. טווח המטרה הוא יחסי באופן הפוך לכוח כמות המחשוב המשותפת של כל רשת הביטקוין בכדי לשמור על זמן ממוצע קבוע למציעת מספר חד פעמי נכון. המשתתף שמצא מספר חד פעמי תקין זוכה באפשרות להוסיף בלוק לבלוקצ'יין; לכן, אלו שמשתמשים ביותר משאבים לחשב את המספר הזה לוקחים תפקיד גדול יותר במצב הבלוקצ'יין. הוכחת עבודה מאפשרת התנגדות נגד התקפת סיביל, שבא יישות מתנהגת כמספר יישויות בכדי להשיג כוח נוסף במערכת מבוזרת וגם בכדי להפחית את מספר מצבי המירוץ שקיימים באופן טבעי במבנה נתונים גלובלי. 

פרוטוקול הסכמה אחר, הוכחת החזקה  )ekatS fo foorP(, הוצג לראשונה על ידי פירקוין )niocreeP( ב2102 \cite{King_peercoin}. במערכת הוכחת החזקה, משתתפים מצביעים עם קול משוקלל השווה לכמות המטבעות שהם מחזיקים. בסידור זה, בעלי הההחזקה הפיננסית הגדולה יותר מקבלים יותר כוח ובאופן טבעי מתומרצים לשמור על כנות המערכת או שיפסידו את השקעתם. הוכחת ההחזקה מבטלת את הצורך בתחרות המבזבזת כוח מחשוב ורק דורשת תוכנה קלת משקל הרצה על חומרה בעלת עוצמה נמוכה.
    
מאמר הנאנו המקורי ויישום הבטא הראשון פורסמו בדצמבר 4102, מה שהופך אותו לאחד המטבעות הקריפטוגרפים הראשונים המבוססים על גרף מכוון ללא מעגלים )\textit{GAD}( \cite{Colin_original_raiblocks}. מיד לאחר מכן, מטבעות קריפטוגרפים מבוססי GAD התחילו בתהליכי פיתוח. הנודעים מביניהם הם דאגקוין/בייטבול ואיוטה \cite{Ribero_dagcoin:a} \cite{Popov_tangle:a}. המטבעות הקריפטוגרפים מבוססי הדאג הללו שברו את תבנית הבלוקצ'ייין ושיפרו ביצועי מערכת ואבטחה. בייטבול משיג הסכמה על ידי הסתמכות על "שרשרת מרכזית" המורכבת מעדים אמינים ובעלי מוניטין גבוה בזמן שאיוטה )ATOI( משיג הסכמה על ידי מערכת הוכחת עבודה על פעולות מצטברות. נאנו משיג הסכמה על ידי הצבעה מאוזנת משקל על פעולות מתנגשות. מערכת הסכמה זו מאפשרת פעולות מהירות והחלטיות יותר ובו זמנית שומרת על מערכת מבוזרת וחזקה. נאנו ממשיך פיתוח זה ומיקם את עצמו כאחד מהמטבעות הקריפטוגרפים בעלי הביצועים הגבוהים ביותר.