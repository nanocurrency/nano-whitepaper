% Status: First Draft; NOT Ready for Review
נאנו, כמו כל מטבע קריפטוגרפי מבוזר, יכול להיות תחת התקפה של גורמים זדוניים למטרות רווח כלכלי או השבתת המערכת. בחלק זה נציין תרחישי התקפה אפשריים, את השלכותיהן של כל התקפה ואיך הפרוטוקול של נאנו לוקח אמצעי הגנה.

\subsection{סינכרון של קפיצת בלוק}
בחלק~\ref{sec:forks}
, דיברנו על תרחיש בו יכול להיות שבלוק לא שודר כהלכה, ובכך גורם לרשת להתעלם מבלוקים עוקבים. אם צומת מבחינה בבלוק שאין לו קישור לבלוק הקודם לו, ישנן שתי אופציות:
\begin{enumerate}
  \item להתעלם מהבלוק מכיוון שהוא יכול להיות בלוק זבל זדוני.
  \item לבקש בקשה לסינכרון מחדש עם צומת אחר.
\end{enumerate}
במקרה של סינכרון מחדש, חיבור PCT חייב להיווצר עם עוד צומת חדש בכדי להתמודד עם כמות התעבורה הגדלה שפעולת סינכרון מחדש דורשת. מצד שני, אם הבלוק אכן היה בלוק זדוני, אז פעולת הסינכרון מחדש היא מיותרת וגורמת לתעבורה גדולה יותר ללא כל צורך. זהו התקפת רשת שגורמת לדחייה של שירות )SoD(.


בכדי להימנע מסינכרונים מחדש מיותרים, צמתים יחכו עד שסף מסוים של קולות נצפה בשביל בלוק שיכול להיות זדוני לפני שהם יתחילו חיבור לצומת חדש בשביל סינכרון. אם הבלוק לא משיג מספיק קולות, אפשר להחשיבו כמידע זבלי.


\subsection{הצפת פעולות}\label{sec:transaction_flooding}
יישות זדונית יכולה לשלוח המון פעולות תקפות אך מיותרות בין חשבונות תחת שליטתה בכדי להרוות את הרשת. מכיון שאין עמלת העברה, אותה יישות יכולה להמשיך את ההתקפה הזו ללא סוף. לעומת זאת, הוכחת העבודה הנדרשת בכל פעולה מגבילה את רצף הפעולות שבה היישות הזדונית יכולה לייצר מבלי להשקיע כספית במשאבי מחשוב. אפילו תחת התקפה כזו שמטרתה להציף את פנקס החשבונות, צמתים שהם לא צמתים עם הסטוריה מלאה יכולים לגזום פעולות עבר מהשרשרת. זה מגביל את השימוש באחסון מסוג זה של מתקפה לכמעט כל המשתמשים.

\subsection{התקפת סיביל}
יישות יכולה להרים מאות צמתים של נאנו על אותה מכונה. לעומת זאת, מכיוון שמערכת ההצבעות היא משוקללת על פי מאזן חשבון, הוספה של צמתים חדשים ברשת לא תוסיף כח הצבעה לתוקף. לכן אין שום תועלת בביצוע התקפת סיביל.

\subsection{התקפת פני-ספנד}
מתקפת פני-ספנד היא מתקפה בה התוקף מבזבז כמויות אינפיניטיסימליות למספר רב של חשבונות בכדי לבזבז את משאבי האחסון של צמתים. קצב פרסום הבלוקים מוגבל על ידי הוכחת העבודה, דבר שמגביל את יצירת החשבונות והפעולות לדרגה מסויימת. צמתים שהם לא צמתים בעלי היסטוריה מלאה יכולים לגזום חשבונות מתחת למדידה סטטיסטית שבה החשבון הוא ככל הנראה לא חשבון תקף. לסיום, נאנו מכוונת לשימוש מינימאלי של אחסון קבוע כך שהגודל הנדרש לשמירה על עוד חשבון הוא יחסי לגודל של בלוק פתוח + מפתוח = $ 96\text{B} + 32\text{B} = 128\text{B}$. זהו שווה ערך לקח שBG1  יכול לאכסן 8 מליון חשבונות פני-ספנד. אם צמתים רוצים לגזום בצורה אגרסיבית יותר, הם יכולים לחשב הפצה שמבוססת על פי כמות כניסות לחשבון ולהעביר חשבונות שאליהם הכניסה נמוכה לאחסון איטי יותר.


\subsection{התקפת הוכחת עבודה מחושבת מראש}
מכיוון שבעל החשבון הינו היישות היחידה שמוסיפה בלוקים לשרשרת החשבון שלו, בלוקים עוקבים יכולים להיות מחושבים, ביחד עם הוכחת העבודה שלהם, לפני שהם משודרים לרשת. כאן התוקף מייצר מספר עצום של בלוקים עוקבים, שבו לכל אחד ערך מינימאלי, במשך תקופה מסויימת של זמן. בשלב מסוים, התוקף מבצע השבתת שירות )SoD( על ידי הצפת הרשת עם המון פעולות תקפות. זוהי גרסא מתקדמת של הצפת הפעולות בחלק~\ref{sec:transaction_flooding}
. כזו התקפה תעבוד רק באופן זמני קצר, אבל יכולה להיות משומשת ביחד עם עוד התקפות כגון התקפת \textless \%05 )חלק~\ref{sec:attack_50}( בכדי להגדיל את הצלחת ההתקפה. הגבלת רצף הפעולות ועוד טכניקות כרגע נחקרות בכדי להתמודד עם התקפה
זו.

\subsection{התקפת \textless \%05} \label{sec:attack_50}
מערכת ההסכמה של נאנו היא מערכת הצבעות מאוזנת משוקללת. אם תוקף יכול להשיג \textless \%05 של כוח ההצבעה, הוא יכול לגרום לרשת להתעלם מהסכמה ובכך לשבור את המערכת.  תוקף יכול להוריד את הכמות של המאזן שהוא צריך להפסיד על ידי חסימה של צמתים טובים מלהצביע דרך שיבוש מערכת )SoD(. נאנו נוקטת באמצעים הבאים בשביל להתמודד עם התקפה זו:
\begin{enumerate}
  \item 
	ההגנה המרכזית נגד סוג זה של התקפה היא העובדה שמשקל ההצבעה שווה להשקעה במערכת. 
בעל חשבון מתומרץ באופן טבעי לשמור על אמינות המערכת כדי להגן על השקעתו. נסיון להפוך את פנקס החשבונות יגרום להרס המערכת כולה, מה שיהרוס את השקעתו.
  
  \item	מחיר ההתקפה הוא יחסי לשווי השוק של נאנו. במערכות הוכחת עבודה, טכנולוגיה יכולה להיות מומצאת כך שתינתן שליטה לא יחסית בהשוואה להשקעה הכספית ואם ההתקפה מצליחה, הטכנולוגיה יכולה לחזור להיות בשימוש אחרי שההתקפה נגמרת. עם נאנו, מחיר התקפת המערכת גדל עם המערכת עצמה ואם התקפה תהיה מוצלחת, ההשקעה בהתקפה לא יכולה לחזור לבעלים.

  \item	בכדי לשמור על קוורום מקסימלי של מצביעים, הצעד הבא של ההגנה הוא הצבעה על ידי נציגים. בעלי חשבון שאינם יכולים להשתתף בהצבעות באופן אמין מסיבות של של חוסר חיבור לרשת יכולים לבחור נציג שישתתף בהצבעה עם משקל המאזן שלהם. בצעד זה אנחנו ממקסמים את מספר הנציגים ואת גיוונם ומחזקים את הרשת.
  
  \item	פיצולים בנאנו אף פעם לא קורים בטעות, אז צמתים יכולים להחליט איך לתקשר עם צמתים מפוצלים. הזמן היחיד בו חשבונות של משתמש לא תוקף חשופים לבלוקים מפוצלים זה אם הם מקבלים מאזן מחשבון תוקף. חשבונות שרוצים להיות בטוחים מבלוקים מפוצלים יכולים לחכות קצת או המון לפני שהם מקבלים מחשבון שמייצר בלוקים מפוצלים או שהם יכולים לבחור לא לקבל לעולם. מקבלים גם יכולים לבחור ליצור חשבונות נפרדים כאשר הם מקבלים כספים ממקורות מפוקפקים בכדי לבודד חשבונות אחרים.
  
  \item	קו הגנה אחרון שעדיין לא נכנס לשימוש הוא מלוט בלוקים )\textit{gnitnemec kcolb}(. נאנו הולך מעל ומעבר בכדי ליישב פיצול בלוקים בצורה מהירה דרך הצבעות. צמתים יוכלו להיות מקונפגים לבלוקי בטון, דבר הימנע מהם לחזור חזרה אחרי תקופה מסויימת של זמן. הרשת היא מספיק בטוחה דרך התמקדות בזמן העברה מהיר בכדי למנוע פיצולים מעורפלים.
\end{enumerate}

גרסא יותר מתוחכמת למתקפה של \textless \%05 מתוארת באיור 9. "מנותקים" הוא אחוז הנציגים שנבחרו אך לא מחוברים לרשת בכדי להצביע. ''החזקה" היא כמות ההשקעה שאיתה התוקף מצביע. "פעילים" היא כמות הנציגים שמחוברים ומצביעים בהתאם לפרוטוקול. תוקף יוכל להוריד את כמות ההחזקה שהם צריכים לוותר עליה על ידי הוצאת מצביעים אחרים מהרשת דרך התקפת SoD. אם התקפה זו יכולה להחזיק, הנציגים שמותקפים יהפכו להיות בלתי מסונכרנים וזה מומחש על ידי ''אסנכרון". לסיום, תוקף יכול להשיג קפיצה קצרה של כוח הצבעה על ידי החלפת התקפת הSoD שלו בקבוצה חדשה של נציגים בזמן שהקבוצה הישנה מסנכרנת מחדש את פנקס החשבונות שלה. זהו מומחש על ידי "התקפה".
\begin{figure}[!h]
\L{
  \centering
  \begin{tikzpicture}[node distance=0.0cm]
    %%%%%%%%%%%%%
    % ACCOUNT A %
    %%%%%%%%%%%%%
    \node (offline) [rec]
        {\R{מנותקים}};
    \node (unsynced) [rec, left = of offline]
        {\R{אסכנרון}};
    \node (attacked) [rec, left = of unsynced]
        {\R{\textbf{התקפה}}};
    \node (active) [rec, left = of attacked]
        {\quad\quad \R{פעילים}\quad\quad\quad};
    \node (stake) [rec, left = of active]
        {\R{החזקה}};
    
  \end{tikzpicture}
  \R{\caption{סידור פוטנציאלי של הצבעה שיכול להוריד \%15 מדרישות התקפה.}}  \label{fig:attack_dist}
  }
\end{figure}

אם תוקף יכול לגרום למצב בו ''החזקה" \textless  "פעילים" על ידי שילוב של כל התרחישים הללו, הוא יוכל להפוך קולות בהצלחה בפנקס החשבונות בתמורה להחזקה שלו. נוכל להעריך כמה סוג זה של התקפה יעלה על ידי בחינה של שווי השוק של מערכות אחרות. אם נניח ש-\%33 מהנציגים מנותקים מהרשת או מותקפים על ידי SoD, תוקף יצטרך לרכוש \%33 משווי השוק בכדי לתקוף את המערכת דרך הצבעה.

\subsection{הרעלת אתחול}
ככל שתוקף מחזיק מפתח פרטי ישן עם מאזן יותר זמן, כך גדל הסיכוי שלמאזן שהיה קיים באותו הזמן לא יהיה נציגים משתתפים בגלל שהמאזן או הנציגים עברו לחשבון חדש יותר. זה אומר שאם צומת מאותחל לייצוג ישן של הרשת איפה שלתוקף יש קוורום של כוח הצבעה בהשוואה לנציגים באותה נקודה בזמן, התוקף יוכל להתעלם מהחלטת הבחירה של צומת זו. אם משתמש חדש זה רוצה לתקשר עם כל אחד חוץ מהצומת התוקף כל הפעולות שלו יידחו בגלל שיש להם ראשי בלוקים שונים. התוצאה הסופית היא שצמתים יכולים לבזבז את זמנם של צמתים אחרים ברשת על ידי שליחה של מידע רע. בכדי למנוע את זה, צמתים יכולים להיות משוייכים למסד נתונים התחלתי של חשבונות ידועים עם ראשי בלוקים טובים; זהו תחליף להורדת כל מסד הנתונים עד לבלוק הראשוני. ככל שההורדה קרובה יותר לזמן הנוכחי, כך גבוה יותר הסיכוי של הגנה מוצלחת מפני ההתקפה. בסוף, התקפה זו היא כנראה לא יותר גרועה משליחת מידע זבלי לצמתים ברגעי האתחול, מכיוון שהם לא יוכלו לייצר פעולות עם כל אחד שיש לו מסד נתונים עכשוי.