% Status: First Draft; Needs more info from dev team.
Currently the reference implementation is implemented in C++ and has been producing releases since 2014 on Github \cite{LeMahieu_github}.

\subsection{Design Features}
The Nano implementation adheres to the architecture standard outlined in this paper. Additional specifications are described here.

\subsubsection{Signing Algorithm}
Nano uses a modified ED25519 elliptic curve algorithm with Blake2b hashing for all digital signatures \cite{Bernstein_ED25519}. ED25519 was chosen for fast signing, fast verification, and high security.

\subsubsection{Hashing Algorithm}
Since the hashing algorithm is only used to prevent network spam, the algorithm choice is less important when compared to mining-based cryptocurrencies. Our implementation uses Blake2b as a digest algorithm against block contents \cite{Aumasson_blake2}.

\subsubsection{Key Derivation Function}
In the reference wallet, keys are encrypted by a password and the password is fed through a key derivation function to protect against ASIC cracking attempts. Presently Argon2 \cite{Biryukov_argon2} is the winner of the only public competition aimed at creating a resilient key derivation function.

\subsubsection{Block Interval}
Since each account has its own blockchain, updates can be performed asynchronous to the state of network. Therefore there are no block intervals and transactions can be published instantly.

\subsubsection{UDP Message Protocol}
Our system is designed to operate indefinitely using the minimum amount of computing resources as possible. All messages in the system were designed to be stateless and fit within a single UDP packet. This also makes it easier for lite peers with intermittent connectivity to participate in the network without reestablishing short-term TCP connections. TCP is used only for new peers when they want to bootstrap the block chains in a bulk fashion.

Nodes can be sure their transaction was received by the network by observing transaction broadcast traffic from other nodes as it should see several copies echoed back to itself.

\subsection{IPv6 and Multicast}
Building on top of connection-less UDP allows future implementations to use IPv6 multicast as a replacement for traditional transaction flooding and vote broadcast. This will reduce network bandwidth consumption and give more policy flexibility to nodes going forward.

\subsection{Performance}
At the time of this writing, 4.2 million transactions have been processed by the Nano network, yielding a blockchain size of 1.7GB. Transaction times are measured on the order of seconds. A current reference implementation operating on commodity SSDs can process over 10,000 transactions per second being primarily IO bound.